\documentclass[12pt]{article}

%Packages
\usepackage[latin1]{inputenc}
% Esto es para que el LaTeX sepa que el texto está en español:
\usepackage[spanish]{babel}
\usepackage[x11names,table]{xcolor}

% Paquetes de la AMS:
%\usepackage[total={6in,11in},top=0.50in, left=1in]{geometry}
\usepackage[top=1in, left=1in, right=1in, bottom=1in]{geometry}
\usepackage{amsmath, amsthm, amsfonts}
\usepackage{graphics}
\usepackage{listings}
\usepackage{float}
\usepackage{epsfig}
\usepackage{amssymb}

\lstset{ %
	language=Python, % lenguaje
	basicstyle=\normalsize\ttfamily,
	keywordstyle=\color{blue},
	commentstyle=\color{blue!50},
	backgroundcolor=\color{gray!9},
	identifierstyle = \color{gray!161},
	stringstyle = \color{yellow},
	numberstyle = \color{green},
	columns=fullflexible,
	showspaces=false
}



\newtheorem{thm}{Teorema}[section]
\newtheorem{cor}[thm]{Corolario}
\newtheorem{lem}[thm]{Lema}
\newtheorem{prop}[thm]{Proposición}
\theoremstyle{definition}
\newtheorem{defn}[thm]{Definicion}
\theoremstyle{remark}
\newtheorem{rem}[thm]{Observación}

\def\RR{\mathbb{R}}

\renewcommand{\labelenumi}{$\bullet$}
\newtheorem{definition}{Definici�n}[section]
\newtheorem{theorem}{Teorema}[section]
\newtheorem{corollary}{Corolario}[section]
\newtheorem{lemma}{Lema}[section]
\newtheorem{proposition}{Proposici�n}[section]
\newcommand{\statement}[3]{
	\begin{center}
		{ \fcolorbox {gray!11}{gray!11}{
				\begin{minipage}[h!]{\textwidth}
					\begin{#1}\label{#3}
						#2
					\end{#1}
				\end{minipage} } }
			\end{center}}
			\renewcommand{\proof}[1]{{\it Demostraci�n}\\ #1 \hfill\blacksquare}
\newcommand{\pagediv}[4]
{
	\begin{figure}[!h]
		\begin{minipage}[b]{#1\textwidth}
			#3			
		\end{minipage} \hfill 
		\begin{minipage}[b]{#2\textwidth}
			#4
		\end{minipage}
	\end{figure}
	
}



%define title
\author{
	Dalianys P\'erez Perera\\
	C-411 
}
\title{Proyecto de L�gica Difusa   \\
	Simulaci�n\\
	}

\date{}
\begin{document}
%generates the title
\maketitle

\selectlanguage{spanish}

\newpage
%insert table contents
\tableofcontents
\newpage
\section{Introducci�n}
Los porblemas que pretenden reflejar un comportamiento medianamente realista suelen requerir un elevado n�mero de reglas mientras que los problemas simples no siempre llegan a tener
aplicaciones en el mundo cotidiano, etc. Con este proyecto se ofrece un ejemplo que muestre el comportamiento de un sistema con reglas b�sicas y sencillas para una aplicaci�n real.

Se ha definido un sistema difuso que controla un sem�foro, del que se describir�n sus entradas, salidas, reglas, operadores de inferencia as� como detalles de implementaci�n.

\section{Caracter�sticas del Sistema de Inferencia}
Se propone un sistema de inferencia compuesto por una base de conocimientos
\section{Principales ideas seguidas para la implementaci�n}
\section{Propuesta de problema}
El modelo se basa en el control de un sem�foro, espec�ficamente la duraci�n de la luz verde, seg�n la cantidad de peatones que necesitan cruzar la calle y la cantidad de carros detenidos en cola durante la luz roja. 
En primer lugar se van a definir las variables de entrada y salida del sistema. El sem�foro tiene dos
entradas y una salida. La primera entrada es la
cantidad de peatones \lstinline|Walkers|. Esta entrada al ser una
variable discreta tiene como dominio el conjunto de
enteros entre 0 y 20. Sobre este dominio se definen
tres conjuntos difusos que definen los predicados: \lstinline|Low|, \lstinline|Medium| y \lstinline|High|. El valor \lstinline|Medium| tiene funci�n de pertenencia trapezoidal y el resto de tipo triangular.

PONER GR�FICO

La segunda entrada tambi�n discreta es la cantidad de tr�fico \lstinline|Traffic|, cuyo dominio est� definido entre 0 y 35. Los conjuntos difusos que esta variable determina son: \lstinline|Null|, \lstinline|Moderate| y \lstinline|Intense|. En el siguiente gr�fico se visualizan los tipos de funciones de pertenencia de cada uno de estos conjuntos.

PONER GR�FICO

La variable de salida es la duraci�n de la luz verde del sem�foro en segundos \lstinline|GreenDuration|, el cual tiene como dominio los reales entre 0 y 100 adem�s de definir tres conjuntos difusos: \lstinline|Short|, \lstinline|Medium| y \lstinline|Long|

Las reglas de inferencia para este sistema difuso est�n simplificadas en la siguiente tabla y sus producciones son del tipo:

 \textit{\lstinline|Walkers| = \lstinline|A| and \lstinline|Traffic| = \lstinline|B| $=>$ \lstinline|GreenDuration| = \lstinline|C| }
 \\
 donde \lstinline|A|, \lstinline|B| y \lstinline|C| son conjuntos difusos definidos anteriormente para cada variable.
 
El comportamiento del sistema difuso depender� de los operadores utilizados durante la fase de \textit{fuzzificaci�n}, de la funci�n de composici�n para computar el consecuente y del m�todo de conversi�n de difuso a n�tido.



CREAR TABLA




\section{Consideraciones obtenidas}


\end{document}
